\documentclass[a4paper]{article}

\usepackage[utf8]{inputenc}
\usepackage[portuges]{babel}
\usepackage{a4wide}
\usepackage{multicol}
\usepackage{spverbatim}
\usepackage{graphicx}

\title{Projeto de SO - Stream Processing\\Grupo 96}
\author{Sérgio Jorge (A77730) \and Vítor Castro (A77870) \and Marcos Pereira (A79116)}
\date{}

\begin{document}

\maketitle

\begin{abstract}
\end{abstract}

\tableofcontents

\section{Introdução}
\label{sec:intro}

Este projeto foi realizado com o objetivo de desenvolver um programa responsável pela conceção de um sistema de \textit{Stream Processing}. Foram, então, propostas pelos professores, a realização de quatro componentes computacionais, construídas em C, que produzem determinadas operações elementares e a realização de um controlador que gere a rede de processamento. A realização das componentes assim como do próprio controlador, permitiram melhorar e consolidar os conhecimentos adquiridos, durante o semestre, na UC de Sistemas Operativos.
Assim, de modo a facilitar a compreensão do projeto, o relatório está dividido da seguinte forma:

\begin{description}
    \item[Secção 2 :] Problema;
    \item[Secção 3 :] Solução;
    \item[Secção 4 :] Conclusão.
\end{description}
\pagebreak

\section{Problema}
\label{sec:problema}
Neste projeto de SO, é-nos pedido para fazermos um sistema de \textit{Stream Processing}. Assim, o que devemos fazer para gerar a rede de processamento são os seguintes módulos:
\begin{description}
\item[1 - const \textless valor \textgreater]\hfill \\
Reproduz as linhas e acrescenta uma nova coluna com o valor que é passado como argumento à função;
\item[2 - filter \textless coluna \textgreater \textless operador \textgreater \textless operando \textgreater]\hfill \\
Reproduz as linhas nas quais a condição dada como argumento se verifica;
\item[3 - window \textless coluna \textgreater \textless operação \textgreater \textless linhas \textgreater]\hfill \\
Reproduz todas as linhas e acrescenta uma nova coluna com o resultado de uma operação sobre os valores da coluna indicada nas linhas anteriores;
\item[4 - spawn \textless cmd \textgreater \textless args...\textgreater]\hfill \\
Reproduz todas as linhas, executando o comando indicado uma vez para cada uma delas e acrescentando uma nova coluna com o valor do exit status;
\item[5 - controlador]\hfill \\
A função do controlador é a de gerir a criação e manutenção de nós e das ligações entre eles.
\end{description}

\section{Solução}
\label{sec:solucao}

\subsection{const}

\begin{itemize}
\item guardamos o valor que é passado como argumento;
\item definimos um array para o qual lemos o que é inserido no stdinput;
\item colocamos o carater "dois pontos" no final do array;
\item inserimos o valor no fim do array;
\item escrevemos o array para o stdoutput.
\end{itemize}

\subsection{filter}

\begin{itemize}
\item guardamos o valor dos argumentos (operador e duas colunas);
\item definimos um array para o qual lemos o que é inserido no stdinput;
\item guardamos os valores que estão nas duas colunas que nos foram indicadas nos argumentos da função;
\item fazemos a comparação entre os valores guardados na alínea anterior;
\item se a comparação for válida, escrevemos a linha para o stoutput. Se não for válida, nada é feito.
\end{itemize}

\subsection{window}

\begin{itemize}
\item guardamos o valor dos argumentos (coluna, operador e linhas);
\item definimos um array A com linhas posições e pomos esse array a zeros;
\item definimos um array B para o qual lemos o que é inserido no stdinput;
\item guardamos o valor que está na coluna do stdinput que nos foi indicada no argumento da função;
\item preenchemos o array A com os valores que vão sendo lidos pela alínea anterior. Este array só tem linhas posições pelo que quando enche, é substituído o valor que está na primeira posição;
\item para as diferentes operações, percorre-se o array A e guarda-se o resultado num array Output;
\item colocamos o carater "dois pontos" no final do array B;
\item inserimos o array Output no fim do array B;
\item escrevemos o array B para o stdoutput.

\end{itemize}

\subsection{spawn}

\begin{itemize}
\item guardamos o valor dos argumentos que têm o carater "\$" na frente para o array A e a posição destes para o array P;
\item definimos um array B para o qual lemos o que é inserido no stdinput;
\item percorremos o array A que contém os valores das colunas a serem lidas e copiamos o que está na coluna para a posição de argv onde o carater "\$" foi encontrado;
\item faz-se um fork que executa através dos argumentos da função que foram entretanto manipulados;
\item guarda-se o resultado da execução num array Output;
\item colocamos o carater "dois pontos" no final do array B;
\item inserimos o array Output no fim do array B;
\item escrevemos o array B para o stdoutput.
\end{itemize}

\subsection{controlador}
O controlador gere a criação de nós e injects, e das ligações entre estes. Para esse efeito, são guardadas informações em 5 arrays globais:

\begin{itemize}
\item nodes - Guarda os process id's de todos os nós em execução

\item connections - Guarda os process id's de todos os processos "ponte" em execução (explicados abaixo)

\item connectionDests - Guarda, relativamente a cada nó, a lista de nós que estão a receber o seu output

\item injects - Guarda os process id's de todos os injects em execução

\item injectConnections - Guarda os process id's de todos os processos "ponte" em execução que ligam um inject a um nó
\end{itemize}

 Por forma a realizar as ligações entre nós, foi necessária a criação de um processo auxiliar, denominado connect. Cada nó tem um destes processos ponte associado a si, que copia o seu output e envia para todos os nós a que está conectado.

 Para tornar possíveis estas ligações foi necessária a criação de um par de named pipes (FIFOs) para cada nó, que substituem o seu stdin e stdout. Assim, um processo ponte consegue introduzir informação em um nó e obter o seu output através destas pipes. Um processo inject tem a si associado apenas uma named pipe, que substitui o seu stdout (um inject nunca recebe dados de outro nó).

 Quando uma das conexões entre nós é modificada, o processo ponte anterior (caso exista) é terminado com SIGTERM (de modo a não serem perdidos dados) e é feito waitpid() para aguardar que termine. De seguida, é usada a informação guardada em connectionDests para juntar os novos ouvintes aos anteriores e criar um novo processo ponte que escreva em todos.

\section{Conclusões}
\label{sec:conclusao}
Este projeto serviu para aprofundarmos o conhecimento da linguagem C, assim como as bibliotecas que lhe estão associadas. Achámos que, com a realização de um trabalho deste tipo permite uma consolidação proveitosa da linguagem, e dos conceitos lecionados nas aulas de Sistemas Operativos. Permite também melhorar as habilidades na resolução de problemas.

\end{document}
